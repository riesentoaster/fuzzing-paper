\documentclass{article}
\usepackage[hidelinks]{hyperref}
\usepackage{csquotes}
\usepackage[vmargin=25mm, hmargin=20mm]{geometry}
\usepackage{longtable}
\usepackage{array}
\usepackage[
    backend=biber,
    sorting=none,
    style=ieee,
    urldate=long,
    maxcitenames=2,
    mincitenames=1
]{biblatex}
\addbibresource{sources.bib}
\usepackage{multicol}
\usepackage{float}
\usepackage{graphicx}

\title{Looking at Challenges and Mitigation in Symbolic Execution Based Fuzzing Through the Lens of Survey Papers}

\begin{document}
% set spacing in itemize to minimum
\let\savedItemize=\itemize
\let\savedEndItemize=\enditemize
\renewenvironment{itemize}{\savedItemize\setlength\itemsep{0px}}{\savedEndItemize}

% disable overfull hbox warnings
\hfuzz=50px

\maketitle
\begin{multicols}{2}
  \tableofcontents

  \section{Introduction}
  In 2022, the cost of poor software quality was estimated to be more than \$2.4 trillion in the US alone.\cite{CostPoorSoftware} Individual cyber attacks have had an estimated financial impact of up to \$8 billion worldwide.\cite{Demystifying} Testing software often accounts for more than 50\% of development costs\cite{Orchestrated}, but it is typically a mostly manual process\cite{PreliminaryAssessment}. Because manual testing requires many developer hours with in-depth knowledge of the system being tested, it does not scale well. Automated testing promises to be more cost effective in finding software defects and has therefore become the most popular vulnerability discovery solution, especially in the industry.\cite{FuzzingASurvey}

  One such automated vulnerability and bug testing technique is fuzzing.\cite{VulnerabilityDiscoveryTechniques} In the seminal work by \citeauthor{UNIX} in \citeyear{UNIX}, the term \textquote{fuzz} is defined as a program that \textquote{generates a stream of random characters to be consumed by a target program}\cite{UNIX}. Since then, a rich ecosystem of fuzzing systems has emerged in both industry and academia, taking design inspiration from a wide variety of software engineering concepts. These are combined into programs that generate various concrete inputs, which are then repeatedly fed into a particular program under test (PUT), and check the program for illegal states or crashes.\cite{EvaluatingFuzzTesting}

  Fuzzing is widely used in industry, with major technology companies and government agencies such as Google, Microsoft, the US Department of Defense, Cisco and Adobe developing proprietary fuzzers and contributing to open source fuzzers. These are then used to great effect, with Google alone finding 20,000 vulnerabilities in Chrome alone using fuzz testing.\cite{Demystifying}

  Another approach to automated software testing is symbolic execution\cite{Symbex}. Test frameworks based on symbolic execution do not run programs with concrete inputs, but with variables representing all possible values. By tracking how these values are used, systems based on pure symbolic execution can then reason about and even prove certain hypotheses in a PUT. However, because these systems must essentially emulate the entire program, including all possible program states, pure symbolic execution only works on trivial programs, and breaks down on real-world programs because of their size. Naïve implementations further cannot handle non-trivial software that may be multi-threaded or interact with its environment.

  Over the past decades, many fuzzers\cite{BitBlaze,CORAL,CREST,CUTE,CVC5,CVCLite,Chopped,Cyberdyne,DART,DTSA,DiSE,DigFuzz,Dowser,Driller,EGT,EXE,Fitnex,FloPSy,GRT,GSE,HCT,HFL,IFL,Intriguer,JFS,KATCH,KLEE,KLEEFP,LATEST,MoWF,Moles,PYGMALION,Pangolin,Pex,QSYM,QuickFuzz,RWset,SAGE,SAVIOR,SMART,SPIN,STP,ScalableAutomatedMethods,TFuzz,TaintScope,VUzzer,SPF} have employed symbolic execution based techniques, with great success: SAGE\cite{SAGE} \textquote{reportedly found a third of the Windows 7 bugs between 2007-2009}\cite{FuzzingTheStateOfTheArt}.

  Since the academic research in this area is vast, existing reviews are used to filter the work on the topic at hand to the most important. The following contributions are made on this basis:

  \begin{itemize}
    \item First, the theoretical principles behind fuzzing and symbolic execution are explained in Section\ref{Theory}.
    \item Second, Section\ref{Methods} explains the reasoning behind using existing survey papers to base this work on.
    \item Third, an overview of existing survey papers investigating the state of fuzzing is given in Section\ref{SurveyPapers}, along with a short summary of each paper.
    \item Fourth, the challenges fuzzing tools face in implementing symbolic execution techniques, and attempts to mitigate each of them, are listed along with examples of works that implement them in Section\ref{Results}.
    \item Finally, limitations and possible additions to this work are discussed in Section\ref{Discussion}.
  \end{itemize}

  \section{Theoretical Principles}
  \label{Theory}

  To understand the limitations of symbolic execution and innovations introduced in papers, a fundamental understanding of both general fuzzing procedures and symbolic execution is crucial.

  \subsection{Fuzzing}

  \citeauthor{UNIX} in \citeyear{UNIX} both invented the term and laid the foundation for fuzzing. They observed that, during a stormy night, lightning strikes would introduce interference in their dial-up based communication channel to a UNIX system, which changed the intended inputs and crashed the tools they were using. They then attempted to reproduce this systematically by repeatedly running tools with random inputs containing a combination of printable, non-printable and \code{NULL} bytes. On different UNIX systems, they were able to crash between 25 and 30\% of all tested utility programs.\cite{UNIX}

  \begin{figure*}[!tp]
    \centering
    \includegraphics[width=0.9\textwidth]{assets/FuzzingSteps.jpg}
    \caption{Architectural Diagram of a Fuzzing System\cite{Science}}
    \label{fig:FuzzingSteps}
  \end{figure*}

  \subsubsection{Similarities Across Fuzzers}
  Ever since then, these fuzzing systems have become more sophisticated and different techniques have been integrated. However, certain characteristics remain similar between all fuzzing systems. They output some concrete input(s) and configurations that can then be used to reproduce the fuzzer's observation, allowing confirmation, reproduction, and debugging of the discovered issues.\cite{EvaluatingFuzzTesting} They automatically find well-defined bugs, such as assertion errors, divisions by zero, \code{NULL} pointer dereferences, etc.\cite{AllYouEverWanted}

  \subsubsection{Architecture of a Fuzzer}
  \label{ArchitectureFuzzer}
  Figure\ref{fig:FuzzingSteps} shows the architecture of a typical fuzzer. Most fuzzers contain similar parts responsible for each step during the fuzzing process. These are explained below

  \paragraph{Target Program}
  The target program (also known as the program under test, or PUT) is the software to be tested. Different fuzzers have different requirements for the kind of PUT they support: Some require access to source code to add instrumentation during the compilation or because they work on a intermediate representation (IR), such as LLVM bytecode.

  \paragraph{Bug Detector}
  The bug detector observes the PUT during execution for illegal states. The simplest implementations trigger on program crashes, more sophisticated bug detectors might check if certain protected parts of the PUT are accessed, for example to check the authentication implemented.

  \paragraph{Bug Filter}
  The list of inputs that put a PUT into an illegal state may contain duplicates, in a sense that the exploit the same software defect. The bug filter attempts to deduplicate this list based on some heuristics like the order basic blocks were executed in or stack hashes.

  \paragraph{Test Case Generator}
  The test case generator's job is to generate and select the next input to be tested. Based on the test case generator, fuzzing systems can be categorized as either mutation-based or generation-based. Mutation-based fuzzers take some input to the PUT as their input, and then repeatedly mutate those if the results produce \textquote{interesting} behavior\cite{EvaluatingFuzzTesting}.

  The prioritization is either done randomly or based on some heuristic, which assign each possible next input a value used to select the next input to be passed to the delivery module. Heuristics often take into account information produced by the monitor like coverage or distance to target instructions. These approaches can further be combined, e.g. by alternating random and heuristic-based input selection.

  \paragraph{Delivery Module}
  The delivery module is responsible to pass the generated test case to the PUT in the expected form and trigger the actual execution. This might be as simple as starting a command line utility with certain command line arguments, but might also include creating files that are accessed by the PUT or even emulating user interaction.

  \paragraph{Monitor}
  Fuzzing systems can further be classified based on how much access the monitor module has to the PUT. Blackbox fuzzers make a single observation: whether or not the PUT crashed. Graybox fuzzers have limited access to the PUT during execution. Typical information extracted by graybox fuzzers includes which basic blocks were executed in what order, or generally code coverage based on instruction, basic block, or statement. Whitebox fuzzers have full access to the PUT and allow for sophisticated reasoning about the program structure. Systems that employ taint analysis or symbolic execution are categorized as whitebox fuzzers, since they deeply examine the program structure during their analysis. Different systems make different tradeoffs, accepting higher analysis cost, in the hope of better bug-finding effectiveness.\cite{EvaluatingFuzzTesting}

  \paragraph{Static Analyizer}
  The static analyzer, as its name suggests, attempts to extract information by statically examining the PUT's source code, IR or binary. Such information might include grammars accepted by the PUT or program paths deemed high-risk.

  \subsection{Symbolic Execution}
  \label{SymbolicExecution}
  The concept of symbolic execution in the context of program testing was introduced in \citeyear{Symbex} by \citeauthor{Symbex}.\cite{Symbex} By not executing a PUT with concrete values (such as the number \code{23} or the string \code{"Hello World!"}) but instead modelling certain or all values with mathematical variables, it allows exploring all possible paths of a program. Different engines (such as angr\cite{angr} or Triton\cite{Triton}) allow performing symbolic execution on source code, an IR, or a binary of a PUT.

  \subsubsection{Performing Symbolic Execution}
  During execution of a certain program, with each instruction a symbolic execution engine computes and updates the so-called symbolic state. It contains:
  \begin{itemize}
    \item the inputs marked as symbolic as symbols $\alpha_i$,
    \item the symbolic expression store $\sigma$, which in turn contains
    \item symbolic expressions $\phi_j$, which are either a reference to a symbol $\alpha_i$, or an arithmetic combination of symbolic expressions, such as $\phi_j=\phi_k-\phi_l$, and finally
    \item the path constraint $\pi$, which is the conjunction of all branch constraints, which are the conditions on symbolic expressions to end up at a certain point in the program, such as $\phi_1\leq2$ or $\phi_2=\phi_3\land\phi_4=3$.
  \end{itemize}

  During execution, the symbolic state is changed according to the specific instruction currently executed. If it performs some form of manipulation on existing data, this is represented by adding new symbolic expressions to the symbolic store. If a branch is (not) taken based on a check on variables or registers containing symbolic expressions, the path constraint is extended with an appropriate condition. The exploration of a program can follow different heuristics, such as breath- or depth-first search. Finally, the path constraint collected along a certain paths can be formulated as a query to a satisfiability modulo theory (SMT) solver. This solver can then prove if the path associated with the constraint is reachable with \textit{any} input, and generate an input that steers the PUT execution exactly the same way as the analyzed execution.

  \subsubsection{Categorizing Symbolic Execution Implementations}
  Symbolic execution implementations can be categorized in several ways: Where static dynamic execution exclusively and exhaustively performs the process described above, dynamic (or concolic, a portmanteau of \textquote{concrete} and \textquote{symbolic}) symbolic execution executes a program symbolically and with concrete values in parallel.

  Online symbolic execution allows calculating multiple paths in parallel, while offline symbolic execution examines one path after the other. Finally, one can distinguish between partial and full symbolic execution, where only a part, or all the variable, and therefore calculation is done symbolically.\cite{Ghidrion}

  \section{Methods}
  \label{Methods}
  Fuzzing is an extensively researched topic. For the search term \textquote{Fuzzing}, Web of Science\cite{WebOfScience} finds 2,741, Scopus\cite{Scopus} 2,410, and Google Scholar\cite{GoogleScholar} approximately 29,300 works. Even when filtered by top venues, to summarize the current state of symbolic execution in fuzzing would be a task beyond what is feasible in this project.

  However, since fuzzing is such a well-published topic, other researchers have taken on the work of summarizing the state of the art, each group with a slightly different focus. These survey papers can therefore be used to approximate a complete picture of the current state of the art. This is the approach chosen by the author of this paper. Section\ref{SurveyPapers} contains a list of survey papers taken into consideration. To ensure accuracy and a consistent level of detail, works discussed in these review papers (primary works) are used to corroborate or refute how review papers discuss the contributions of each primary work.

  To select review papers to consider, the following rules were applied:
  \begin{itemize}
    \item First, different search engines were used to create a list of survey papers in the field of fuzzing. Since even the amount of survey papers is overwhelming (Google Scholar\cite{GoogleScholar} returns more than 25,000 results for the search term \textquote{fuzzing survey}), works were only included if their title, abstract, or keywords contained the word section \textquote{fuzz} and if Google Scholar\cite{GoogleScholar} reports that they were cited more than 5 times or if they were published in a high-impact venue.
    \item Then, further review papers discussed, listed, or cited in other review papers were added (such as works listed in \cite{Demystifying}).
    \item Works published before January 1, 2010 (such as\cite{ViolatingAssumptionsWithFuzzing} or\cite{NewTrendsSymbex}) were disregarded to further limit the scope of this survey.
    \item Then, review papers focusing on a specific technique (such as machine learning\cite{ML1, ML2}) were disregarded.
    \item Similarly, review papers focusing on hybrid fuzzing\cite{Hybrid, Exploratory} were discarded, since most hybrid fuzzers only use symbolic execution in a very limited fashion, which negates most obstacles classic symbolic execution based fuzzers face. To still get some introductory insight, one short survey paper about hybrid fuzzing\cite{SurveyHybrid} was selected to be included.
    \item Additionally, works focusing on a specific use case for fuzzing, such as testing internet of things or other embedded devices\cite{IoT, Embedded, Embedded2}, network protocols\cite{Network, Network2023}, smart contracts\cite{Ethereum}, or JavaScript engines\cite{JavaScript, JavaScript2} were disregarded.
    \item Finally, a selection of additional works were added to the list based on the author's intuition.
  \end{itemize}

  \section{Related Work}
  \label{SurveyPapers}

  This section summarizes contributions of existing survey papers selected as described in Section\ref{Methods} and lists relevant primary works discussed in each. Primary works mentioned without discussion are omitted.

  \papertitle{AllYouEverWanted}
  Using a simple intermediate language (SIMPIL), this paper discusses taint analysis and forward symbolic execution, including examples and analysis of the theoretical foundations of symbolic execution. While fuzzing is mentioned in multiple instances, it is not the main focus. However, it still lists many of the drawbacks and advantages fuzzers based on symbolic execution have, and the additional perspective was valuable in assembling this review.

  \papertitle{PreliminaryAssessment}
  After giving a short overview of issues faced by symbolic execution based fuzzers, this paper focuses on eight high impact fuzzing tools (JPF-SE and Symbolic (Java) PathFinder\cite{JPFSE, JavaPathFinder}, DART\cite{DART}, CUTE\cite{CUTE} and jCUTE\cite{ExplicitPathModelChecking}, CREST\cite{CREST}, SAGE\cite{SAGE}, Pex\cite{Pex}, EXE\cite{EXE}, and KLEE\cite{KLEE}).

  \papertitle{FuzzingTheStateOfTheArt}
  \citeauthor{FuzzingTheStateOfTheArt} from Australia's Department of Defence provide an extensive look at fuzzing — \citetitle{FuzzingTheStateOfTheArt} is the longest of the discussed survey papers. After discussing the taxonomy, concepts, types, and history of fuzzing, they discuss a list of 15 works from scientific literature and ten commercial and open-source frameworks. In those scientific works, they present four papers that employ symbolic execution, namely KLEE\cite{KLEE}, SAGE\cite{SAGE}, GWF\cite{GWF}, and TaintScope\cite{TaintScope}.

  \papertitle{ReviewThreeDecades}
  This survey paper, as the title suggests, focuses on symbolic execution. Starting with an explanation of classical symbolic execution, it then provides a list of issues that fuzzing tools based on symbolic execution face, along with attempts to mitigate those by adapting and extending the algorithms. Finally, the authors present five high-impact tools they worked on: DART\cite{DART}, CUTE\cite{CUTE}, CREST\cite{CREST}, EXE\cite{EXE}, and KLEE\cite{KLEE}.

  \papertitle{Orchestrated}
  Orchestrated surveys \textquote{consist of a collaborative work collecting self-standing sections, each focusing on a key surveyed topic}\cite{Orchestrated}. One of the topics discussed in this particular work is symbolic execution. It contains short introductions into BBRBP\cite{BBRBP}, AGLT\cite{AGLT}, Darwin\cite{Darwin}, MATRIX RELOADED\cite{MATRIXRELOADED}, and SRA\cite{SRA} in its introduction. As mitigation strategies for path explosion, it discusses SMART\cite{SMART}, DDCSE\cite{DDCSE}, PFA\cite{PFA}, among others. For environment interaction, ASSIE\cite{ASSIE} and Cinger\cite{Cinger} are presented. Finally, to attempt to solve constraints that are too complex for direct SMT solver invocation, MCSS\cite{MCSS}, CORAL\cite{CORAL}, and its extension introducing AVM\cite{CORALAVM} are cited.

  \papertitle{Science}
  After discussing the structure of fuzzing systems and different targets observed in literature and industry, this paper focuses on inventions of primary works along the structure of fuzzers. Symbolic execution is discussed in the context of the sample generator and the monitor (see Section\ref{ArchitectureFuzzer}). Discussed primary works include CUTE\cite{CUTE}, KLEE\cite{KLEE}, SAGE\cite{SAGE}, TaintScope\cite{TaintScope}, BuzzFuzz\cite{BuzzFuzz}, GWF\cite{GWF}, Dowser\cite{Dowser}, BORG\cite{BORG}, Driller\cite{Driller}, and MoWF\cite{MoWF}. It further contains a sizeable section comparing symbolic execution based fuzzing systems and their contributions.

  \papertitle{FuzzingASurvey}
  \citeauthor{FuzzingASurvey} focus on coverage-guided fuzzing, mentioning other approaches that can be mixed in and different applications it can be used for. They further broadly discuss the challenges symbolic execution in fuzzing faces. Last, they present TaintScope\cite{TaintScope} and Driller\cite{Driller} as examples of using symbolic execution for specifically for path exploration.

  \papertitle{FuzzingStateOfTheArt2018}
  The first mention of a symbolic execution based fuzzer in this paper is SAGE\cite{SAGE}, which the authors use to summarize the limitations of whitebox fuzzing. In the following section, works bringing progress to a step in the fuzzing workflow are discussed, including SYMFUZZ\cite{SYMFUZZ}, TaintScope\cite{TaintScope} and KATCH\cite{KATCH}. Then, and most clearly relevant for this work, \citetitle{FuzzingStateOfTheArt2018} contains a discussion of the limitations of taint analysis and dynamic symbolic execution. Discussed works in this section include, among others,  SMART\cite{SMART}, HOTG\cite{HigherOrderTestGeneration}, Cloud9\cite{Cloud9}, APLS\cite{APLS}, SAGE\cite{SAGE}, CGF\cite{CGF}, DeepFuzz\cite{DeepFuzz}, TCR\cite{TCR}, DiSE\cite{DiSE}, and CRAXfuzz\cite{CRAXfuzz}.

  \papertitle{EvaluatingFuzzTesting}
  While not a classic survey paper, \citetitle{EvaluatingFuzzTesting} finds issues in how all 32 papers performed the evaluation of the system they introduced. It further proposes rules to follow to make an evaluation robust. Last, it contains a list of what advances each paper examined claims to introduce.

  \papertitle{HackArtScience}
  \citeauthor{HackArtScience} gives an easy-to-read introduction to fuzzing in this work. He distinguishes between blackbox, grammar-based, whitebox, greybox, and hybrid fuzzing and provides code examples that show how some of these approaches work. In the section about whitebox fuzzing, SAGE\cite{SAGE} is discussed comparatively extensively, KLEE\cite{KLEE}, S2E\cite{S2E}, and SPF\cite{SPF} are mentioned.

  \papertitle{SurveyHybrid}
  This fairly short paper focuses, as the title would suggest, on hybrid fuzzing — the combination of black-/greybox and whitebox fuzzing. Discussed work includes Hybrid Fuzz Testing\cite{HybridFuzzTesting}, Driller\cite{Driller}, QSYM\cite{QSYM}, SAVIOR\cite{SAVIOR}, and Pangolin\cite{Pangolin}.

  \papertitle{ChallengesAndReflections}
  Compared to other works, the authors of this article pursue a less technical and more conceptual approach to surveying the state of the art in fuzzing. They identify improvement areas such as usability, residual risk estimation, and fuzzer evaluation techniques and discuss current approaches and their limitations. To further legitimize their findings, they conducted a survey under security professionals from both academia and industry. The only symbolic execution based fuzzers mentioned are KLEE\cite{KLEE}, SAGE\cite{SAGE}, and Mayhem\cite{Mayhem}.

  \papertitle{ArtScienceEng}
  Starting with proposing a taxonomy for fuzzing itself and categorizing fuzzers, this paper proposes a general-purpose model of fuzzing, explaining the steps and approaches common fuzzers share. It further presents a genealogy, tracing the origins of important papers back to the work of \citeauthor{UNIX}. However, it \textquote{does not provide a comprehensive survey on DSE}\cite{ArtScienceEng}, but only discusses whitebox fuzzing in a subsection and refers to other survey papers such as \cite{Orchestrated, AllYouEverWanted} for a more complete overview.

  \papertitle{FuzzingASurveyforRoadmap}
  Similar to what is attempted in this paper, \citetitle{FuzzingASurveyforRoadmap} lists issues along common steps in fuzzing along with attempted solutions, but without the focus on symbolic execution. It does contain a short section about symbolic execution in the context input search space handling, but only discusses very few papers directly while often mentioning entire families of papers, with only some relying on symbolic execution.

  \papertitle{FuzzingVulnerabilityDiscoveryTechniques}
  After a short chapter on fuzzer classification, the main focus of this paper are steps and issues along a typical fuzzer workflow, told through the papers that made advances in each category. Finally, it presents current challenges in research and how they could be approached. The discussed papers include some that rely on symbolic execution: Angora\cite{Angora}, T-FUZZ\cite{TFuzz}, MoWF\cite{MoWF}, and HFL\cite{HFL}.

  \papertitle{SystematicReview2023}
  The authors of this paper guide the reader through advances in fuzzing along the works that introduced those. It includes a section about symbolic execution, which considers the following systems: Driller\cite{Driller}, QSYM\cite{QSYM}, SAVIOR\cite{SAVIOR}, DigFuzz\cite{DigFuzz}, Pangolin\cite{Pangolin}, and QuickFuzz\cite{QuickFuzz}.

  \papertitle{Demystifying}
  This paper dedicates one of its chapter to first explaining the fundamental logic of symbolic execution, and then presenting three implementations (Driller\cite{Driller}, CONFETTI\cite{CONFETTI}, and FUZZOLIC\cite{FUZZOLIC}). It further investigates advances in IoT firmware and kernel fuzzers, but does not explain where up- and downsides of using symbolic execution in these domains lay.

  \section{Challenges and Mitigation}
  \label{Results}
  This section focuses on attempts to mitigate inherent issues with symbolic execution. It categorizes the challenges and presents a (inexhaustive) list of works that introduced some improvement to deal with them. Many of the listed papers implemented further more general efficiency improvements, like SAGE's\cite{SAGE} (and multiple other papers inspired by it) generational search, which generates multiple new inputs from just one run of the symbolic execution engine by solving the constraint formula with the constraint from each branch flipped independently.

  \subsection{Environment Interactions}
  Within a symbolic execution environment, one can reason about program behavior, but this obviously breaks down when a program includes actions that interact with the real world and can not be modelled. Examples of this would be unhandled instructions, system calls, interrupts, inter-process communication like pipes or sockets, or interaction with external systems in general, since they might return unpredictable results because their logic is opaque to the symbolic execution engine.\cite{Demystifying} They might further introduce additional symbolic variables in their return values or have other side effects.

  \paragraph{Concolic Execution}
  Because of this, almost all systems examined perform some form of concolic execution. This allows them to simply use the concrete values whenever they encounter instructions that cannot be executed symbolically, acting as an exit whenever no other approach solves an issue. By using concrete values however, symbolic execution systems sacrifice completeness. But for any system that interacts with the environment in non-trivial ways, this usually is the only way to still test them. For non-emulatable function calls, parameters that are not influenceable by inputs — i. e. they are not symbolic — are just used directly, while values for symbolic variables are chosen randomly from the set of solutions to the current constraints on them.\cite{PreliminaryAssessment} Additionally, there are further advances that certain systems have made, which mean they do not have to drop down to concrete values but can evaluate more logic symbolically.

  \paragraph{Emulating Function Calls}
  One attempt to deal with system calls that is common across symbolic execution engines is to not perform system calls or even call to external libraries, but present the symbolic execution engine with code emulating their effects. This preserves the logical integrity of the resulting constraints and is often more efficient than the original library since it can rely on the operators understanding of the function's effect and does not rely on ways this logic has to be transformed to be executable by a machine. Examples for this technique would be EXE\cite{EXE} or KLEE\cite{KLEE}, and all other systems building on top of either of those (including PYGMALION\cite{PYGMALION}, KATCH\cite{KATCH}, and Cloud9\cite{Cloud9}).

  The downside of this approach is obvious: These summaries have to be created, maintained, and tested manually, while fuzzing systems generally strive to need as little operator interaction as possible. There are attempts to automatically infer input-output relational logic in the form of path constraint to be added for complex instructions based on executing it with many different values\cite{ASSIE}. Additionally, there are systems that analyze a PUT and prompt the operator to present models only for the program parts that actually introduce imprecision, like Cinger\cite{Cinger}.

  \paragraph{Kernel Fuzzing}
  HFL\cite{HFL} is a kernel fuzzer that heavily relies on symbolic execution. It lists three main issues the authors had to overcome: \textquote{(1) indirect control transfers determined by system call arguments (2) controlling and matching internal system state via system calls, and (3) nested argument type inference for invoking system calls}\cite{HFL}. To solve those issues, HFL \textquote{(1) converts implicit control transfers to explicit transfers, (2) infers system call sequence to build a consistent system state, and (3) identifies nested arguments types of system calls}\cite{HFL}.

  \subsection{Memory Modelling}
  As described in Section\ref{SymbolicExecution}, symbolic execution engines keep a memory representation of the process the PUT is running in in memory. However, modelling this memory poses a few challenges.

  \paragraph{Arithmetic}
  Operations on a physical machine differ from pure mathematical operations on symbolic expressions. For example, integers might over- or underflow. Modelling floating point arithmetic is even more difficult, since the imprecision introduced by rounding needs to be represented exactly in the resulting constraints to ensure accuracy. These issues are widely solved by symbolic execution engines by not representing numbers as mathematical numbers, but as bitmaps. The appropriate constraints then also need to be added on a bitmap level, thus introducing additional complexity. Certain engines further employ special constraint solvers that improve floating point based constraint handling (like FloPSy\cite{FloPSy}) and complex mathematical constraints (like CORAL\cite{CORAL} and its extension\cite{CORALAVM}).

  \paragraph{Symbolic Pointers}
  If an instruction performs a jump to a symbolic address, this presents a unique challenge to symbolic execution units, since this single operation increases the size the program's state space. This is a major scaling problem to the point of making calculations on any but the most trivial programs infeasible. The easiest, yet most inaccurate, way to deal with this challenge is to simply drop down to a concrete value, as described above, with the discussed downside of lost completeness. To maintain more than this base-level of accuracy, several strategies have been proposed.

  First, one can separate different operations on symbolic pointers. While dereferencing a symbolic pointer poses a challenge, comparing pointers might still be doable. Early adopters of this strategy included CUTE\cite{CUTE} and CREST\cite{CREST}, which only considers (in-)equality predicates with symbolic pointers. Both static analysis and solving the constraints on the symbol in question can potentially drastically reduce the amount of valid values it can take, thus reducing the amount of states needed to explore.

  EXE\cite{EXE} and KLEE\cite{KLEE} introduced the concept of regarding symbolic pointers as array accesses. An object accessed with a symbolic pointer is copied as often as necessary to model all possible results, including error states. Or, in other words, \textquote{a sound strategy is to consider it a load from any possible satisfying assignment for the expression}\cite{AllYouEverWanted}.

  While this works well if only one symbolic pointer is used, the situation becomes more complex once different symbolic pointers access the same data type from different parts of the program. This is because they might be accessing different or the same values, thus again creating a scaling problem by introducing many permutations. To deal with this issue (called memory aliasing), symbolic execution engines might perform alias analysis at run-time as in DART\cite{DART}. Other systems like Vin\cite{BitBlaze} optionally rewrite all memory addresses as scalars based on their name. While this is efficient, it is based on a potentially unsound assumption of variable name to value equality. Finally, certain SMT solvers (like STP\cite{STP} or Z3\cite{Z3}) can handle alias analysis, and can be used to delegate (some of) the work. This is done by, among others, EXE\cite{EXE}, KLEE\cite{KLEE}, and SAGE\cite{SAGE}.

  \subsection{Path Explosion}
  One of the primary issues symbolic execution faces is the so-called path explosion. This refers to the fact that the program path count is usually exponential in the number of static branches in the code. Because naïve symbolic execution attempts to emulate all possible branches, this requires an unobtainable amount of memory and compute resources for all but trivial programs. If examination is performed based on a simple depth-first search, it gets stuck in non-terminating loops with symbolic conditions and is therefore rarely used. Both EXE\cite{EXE} and KLEE\cite{KLEE} can however be configured to run in this mode.

  Symbolic execution, if implemented this way, can inherently help with path explosion, by only examining possible branches. An example: When running EXE\cite{EXE} on \code{tcpdump}, only 42\% of instructions contained symbolic operands, less than 20\% of of symbolic branches have both sides feasible.\cite{EXE}

  One simple approach is to set a user-defined timeout for the symbolic execution engine, after which approaches not based on symbolic execution are used. This is called hybrid fuzzing\cite{HybridFuzzTesting}. It is further discussed in Section\ref{HybridFuzzing}.

  Other than these, three main strategies are common: Reducing the search space, usage of advanced data structures and other optimizations to delay path explosion, and guiding the search to limit the effects of path explosion.

  \subsubsection{Search Space Reduction}

  Search space reduction can be performed in a few ways: First, \textquote{if a program path reaches the same program point with the same symbolic constraints as a previously explored path, then this path will continue to execute exactly the same from that point on and thus can be discarded.}\cite{RWset} This makes use of the fact that there are often multiple ways to get to the same program state by performing symbolic state analysis at runtime. Further optimizations on the basis of partial order and symmetry reductions were introduced by \citeauthor{GSE}.\cite{GSE}

  Similarly, MoWF\cite{MoWF} uses knowledge gained by its built-in blackbox fuzzer to prune invalid inputs and thus prevents its symbolic execution engine to get stuck in input checking and error handling code. CESE\cite{CESE} uses context-free grammars to limit its symbolic execution engine to interesting paths, as opposed to error handling during parsing. TCR\cite{TCR} intelligently reduces existing test cases and prioritizes the remaining according to heuristics to maximize exploration efficiency.

  \subsubsection{Using Advanced Data Structures}

  \paragraph{State Represantation} Since the symbolic state often only has minor differences between closely related paths, it can be represented by a data structure that allows copy-on-write. This means that only the changed parts actually need to be stored, reducing memory consumption significantly. Many systems employ this strategies: KLEE\cite{KLEE}, Mayhem\cite{Mayhem}, S2E\cite{S2E}, BORG\cite{BORG}, and Cloud9\cite{Cloud9}. States can also be merged statically, like in KLEE-FP\cite{KLEEFP}. To further lower pressure on memory, state information can be transferred to disk, like in Mayhem\cite{Mayhem}, BORG\cite{BORG}, and SAGE\cite{SAGE}. The latter also introduced a compact representation for path constraints.\cite{SAGE}

  \paragraph{Function and Data Structure Summaries} To limit the size of paths that need to exercised (and thus reducing the amount of symbolic expressions and the size of path constraints), functions can be analyzed and represented by a summary. These summaries can either be generated automatically, like in SMART\cite{SMART}, Higher Order Test Generation\cite{HigherOrderTestGeneration}, or Demand-Driven Compositional Symbolic Execution\cite{DDCSE}, or manually, as described by \citeauthor{PFA}\cite{PFA}. The latter also introduced a system where common complex structures like strings and regular expressions can be manually transformed into constraints.\cite{PFA} Using function summaries essentially reduces the problem of interprocedural paths to reasoning about intraprocedural paths. This was demonstrated in LATEST\cite{LATEST}.

  \subsubsection{Guiding the Execution}
  The approach to mitigating the path explosion problem receiving the most attention is execution guiding. It attempts to steer the execution to parts of the PUT that are interesting in an attempt to finish analysis (or at least produce findings) as soon as possible, thus reducing execution time before findings are produced to a bearable minimum.

  Since it is unknowable where exactly findings will be produced, search heuristics of varying complexity are employed to choose the next input to evaluate. These might take any number of inputs and trade off increased complexity for better precision.

  \paragraph{Dynamic Analysis} Most commonly, symbolic execution based fuzzers guide the execution by using the input which increases code (instruction, basic block, or statement) coverage. Examples include EXE\cite{EXE}, SAGE\cite{SAGE}, and CREST\cite{CREST}. Complementary, Fitnex\cite{Fitnex} is a state-dependent fitness function that measures distance between already discovered feasible paths is to a particular test target (as defined by any other heuristic).

  Other systems reward inputs that lead to longer runtime, (like in Automatic Generation of Load Tests\cite{AGLT}), or those that produce vastly different outputs based on very similar inputs, as in Symbolic Robustness Analysis\cite{SRA}. QuickFuzz\cite{QuickFuzz} weighs the approximate cost of executing a certain path against its demand and selects the next input based on this ratio.

  \paragraph{Static Analysis} The second group of attempts revolves around static analysis, usually in the form of examining the control flow graph (CFG) of a PUT. CREST\cite{CREST} \textquote{is an extensible platform for building and experimenting with heuristics for selecting which paths to explore}\cite{ReviewThreeDecades} and supports analysis of CFGs. This analysis might mark program parts as interesting based on different criteria and guide the execution towards these. GRT\cite{GRT} and VUzzer\cite{VUzzer} are examples of systems using general static analysis. Examples of more specific static analysis heuristics include the following:

  \begin{itemize}
    \item Directed Incremental Symbolic Execution\cite{DiSE}, Directed Test Suite Augmentation\cite{DTSA}, MATRIX RELOADED\cite{MATRIXRELOADED}, and KATCH\cite{KATCH} guide execution towards code that changed in a patch, thus using fuzzing as regression testing.
    \item Chopper\cite{Chopped} also uses code differences between two versions of a PUT to guide execution, but inverts the incentive structure: It ignores (resp. only lazily executes) certain functions deemed uninteresting to focus on certain parts of the PUT and prevent path explosion. While this designation could be done manually, Chopper\cite{Chopped} uses a heuristic which marks as uninteresting parts of the PUT which are far away from code changed in the examined patch.
    \item CRAXfuzz\cite{CRAXfuzz} includes heuristics to determine interesting function calls like \code{malloc}.
    \item Dowser\cite{Dowser} looks for pointer dereferences in loops.
    \item SAVIOR\cite{SAVIOR} employs UndefinedBehaviorSanitizer\cite{UndefinedBehaviorSanitizer} to find potential bugs.
    \item BORG\cite{BORG} guides analysis towards buffer over-reads.
  \end{itemize}

  Alternatively, the approach can be inverted where execution is guided backwards from interesting parts of the PUT towards outer control structures, such as in DrillerGo\cite{DrillerGo}.

  \paragraph{Probabilistic Approaches}
  More recently, works have introduced approaches which employ heuristics to assign a probability to each potential next input and select which one to execute based on this value. DeepFuzz\cite{DeepFuzz} does this with the goal of finding inputs which will penetrate deeper into a PUT. MEUZZ\cite{MEUZZ} employs machine learning to examine potential next inputs and uses symbolic execution to validate the results. Finally, DigFuzz\cite{DigFuzz} uses Monte Carlo path optimization to quantify the difficulty of each path using grey-box fuzzing and then lets the white-box fuzzer focus on the paths that are believed to be most challenging for grey-box fuzzing to make progress.


  \subsection{Constraint Solving}
  \label{ConstraintSolving}
  Dominates runtime
  \begin{itemize}
    \item Irrelevant constraint elimination: Generally, we go from a solvable constraint set (namely the current execution with the solution being the current concrete values) to one where only one constraint changes (the one we flipped). Typically, major major parts of the constraint set are not influenced by the change and can be excluded from what is passed to the solver. We can then just use the values from the previous iteration. This is implemented, among others, in \cite{SAGE}.
    \item Concolic/dynamic symbolic execution, use concrete values, as in DART\cite{DART} or CUTE\cite{CUTE}.
    \item Similarly, if the whole constraint is not solvable, solve parts of it and use those concrete values to solve the rest, as in MCSS\cite{MCSS}.
    \item Identify independent sub-queries and solve them independently, as is done in EXE\cite{EXE} and KLEE\cite{KLEE}.
    \item \textquote{loop-guard pattern matching rules to identify a constraint that defines the number of iterations of input-dependent loops during dynamic symbolic execution, then set new constraints representing the pre- and post-loop conditions to summarize sets of executions of that loop}\cite{Science}, as in SAGE\cite{SAGE}, BORG\cite{BORG}, and APLS\cite{APLS}
    \item Optimizing SMT queries before passing them to the solver. The optimization itself, however, can already be too complex to compute to employ this strategy effectively.
    \item Mocking and stubbing: Moles\cite{Moles}
    \item Incremental solving: Reuse the results of previous similar queries, because subsets of the constraints are still solved by the same results and supersets often do not invalidate existing solutions. (CUTE\cite{CUTE} and counterexample caching scheme in KLEE\cite{KLEE})
    \item Cache prior SMT query results and reuse them for future queries. Pangolin\cite{Pangolin} uses polyhedral path abstraction to replace query parts for more efficient models based on prior results.
    \item Improved SMT solvers (like Z3\cite{Z3} used in e.g. SAGE\cite{SAGE}, STP\cite{STP}, Yices\cite{Yices}, or cvc5\cite{CVC5} built during the development of EXE\cite{EXE})
    \item Intriguer\cite{Intriguer} uses taint analysis to discover instructions accessing a wide range of input bytes, and then performs symbolic execution for those instructions deemed important and only invoke the underlying SMT solver for complicated queries.
    \item SMT formulas can be transformed into programs, which in turn can then be solved using a coverage-guided fuzzer to generate solutions to the initial formula. JFS\cite{JFS} uses this technique to find solutions to floating-point constraints.
    \item Do not use intermediate representation (IR) to execute symbolically, but integrate the symbolic emulation with the native execution through dynamic binary translation, which prevents additional instructions (since often multiple RISC instructions are necessary to replace one CISC instruction), and allows finer-grained control over the constraint, thus making it smaller. Example: QSYM\cite{QSYM}
    \item While technically not using symbolic execution, analyzing the dependency between input data and state space and approximating solutions to path constraints based on those to prevent expensive calls to an SMT solver is fairly common\cite{WEIZZ}:
          \begin{itemize}
            \item Eclipser\cite{Eclipser} uses instrumentation on the PUT to generate partial path conditions, which can then be solved without invoking SMT solvers to generate further inputs.
            \item Angora\cite{Angora} uses solves path constraints by using a combination of context-sensitive branch coverage, scalable byte-level taint tracking, gradient descent searching, input length exploration, and type and shape inference.
            \item REDQUEEN\cite{REDQUEEN} exploits the fact that much of the input data ends up in the state-space and uses simple transformations on the input data as opposed to relying on taint analysis or symbolic execution to bypass checksums.
            \item Other proposals include approximate SMT solvers, such as FUZZY-SAT implemented in FUZZOLOGIC\cite{FUZZOLIC}, or optimistic SMT solvers, as in QSYM\cite{QSYM}.
          \end{itemize}
  \end{itemize}

  \subsection{Handling Concurrency}
  \begin{itemize}
    \item Testing usually difficult because of the inherent non-determinism.
    \item \textquote{Concolic testing was successfully combined with a variant of partial order reduction to test concurrent programs effectively.\cite{ScalableAutomatedMethods, OpenDistributedPrograms, ExplicitPathModelChecking,RaceDetection}}\cite{ReviewThreeDecades}
    \item \textquote{Generalized Symbolic Execution\cite{GSE} performs symbolic execution by leveraging an off-the-shelf model checker, whose built-in capabilities allow handling multi-threading (and other forms of non-determinism)}\cite{PreliminaryAssessment}
    \item \textquote{This method requires that a sequential version of the program be provided, to serve as the specification for the parallel one. The key idea is to use model checking, together with symbolic execution, to establish the equivalence of the two programs.}\cite{SPIN} (complex parallel numerical computations)
  \end{itemize}

  \subsection{Recursive Data Structures}
  \begin{itemize}
    \item \textquote{GSE handles input recursive data structures by using lazy initialization. GSE starts execution of the method on inputs with uninitialized fields and non-deterministically initializes fields when they are first ac- cessed during the method's symbolic execution.}\cite{PreliminaryAssessment}
    \item \textquote{Pex\cite{Pex} supports the generation of test inputs of primitive types as well as (recursive) complex data types, for which Pex automatically computes a factory method which creates an instance of a complex data type by invoking a constructor and a sequence of methods, whose parameters are also determined by Pex.}\cite{PreliminaryAssessment}
  \end{itemize}

  \subsection{Selective Symbolic Execution}
  \label{HybridFuzzing}
  Not exhaustive.
  \begin{itemize}
    \item Tools like Driller\cite{Driller} perform classical fuzzing (in the case of Driller using AFL\cite{AFLPlusPlus}) until they are \textit{stuck}, meaning they are unable to produce inputs that discover additional paths. Driller then, and only then, invokes its concolic execution engine (Driller uses angr\cite{angr}) to trace the program under investigation executed with one of the previously generated inputs. Finally, it solves the resulting path constraints with one condition flipped to produce an input that will reach new parts of the software. Because Driller does not use symbolic execution for its primary discovery tool, it does not suffer from issues such as path explosion, because it only ever executes one path at a time using symbolic execution. DrillerGo\cite{DrillerGo} uses a similar approach, but targets \textquote{unsafe system calls or suspicious locations, or functions in the call stack of a reported vulnerability that we wish to reproduce}\cite{DrillerGo}. Other targeted hybrid fuzzing systems include 1dVul\cite{1dVul} or BugMiner\cite{BugMiner}.
    \item Interleave random and symbolic execution. Examples: Hybrid Concolic Testing\cite{HCT, Driller, Cyberdyne}
    \item By removing code blocks that are deemed irrelevant, T-FUZZ\cite{TFuzz} prevents its mutation-based fuzzer (which does not use symbolic execution) from getting stuck. It then employs symbolic execution to validate the bugs found.
    \item IFL\cite{IFL} generates quality input to a smart contract based on a symbolic execution engine and then uses them to train a neural network. This can then be used to fuzz other smart contracts since they often implement similar functionality.
    \item TaintScope\cite{TaintScope} uses taint analysis to bypass checksum checks and then symbolic execution to fix checksum fields in malformed test cases.
    \item Similary, Stinger\cite{Stinger} analyses a PUT and then only employs symbolic execution on the parts of the PUT that contribute to path constraints. Like this, the overhead introduced by a symbolic execution engine can be limited to parts of the program.
  \end{itemize}

  \subsection{Using Symbolic Execution in Other Contexts}
  \begin{itemize}
    \item PYGMALION\cite{PYGMALION} uses symbolic execution to generate a grammar from a program, generates valid inputs from that, and finally uses those in fuzzers (AFL\cite{AFLPlusPlus} and KLEE\cite{KLEE}) to measure the achieved code coverage. AUTOGRAM\cite{AUTOGRAM} accomplishes a similar goal of producing a context-free grammar a PUT accepts by executing it with different inputs based on taint analysis.
    \item SYMFUZZ\cite{SYMFUZZ} uses symbolic execution to determine the optimal mutation ratio from a given program-seed pair.
    \item If a program by a software vendor crashes an a customer's device, BBRBP\cite{BBRBP} generates new inputs that do not or at least to a lesser extent contain private information. To achieve this, the instructions executed if a program is fed the initial crashing input are recorded, and then run using symbolic execution. Finally, the generated path constraint is solved by an SMT solver, which produces a new, unrelated input.
    \item Darwin\cite{Darwin} uses symbolic execution to produce inputs that differ slightly from crashing inputs, which can then be used to triage and debug a certain software defect.
  \end{itemize}

  \section{Discussion and Future Work}
  \label{Discussion}
  \subsection{Future Work}
  This work marks only the beginning of what information can be extracted from the combination of review papers and primary works discussed in this work or even in this field in general. What follows is a list of ideas for further analyses.

  \subsubsection{Additional Review Papers to Consider}
  Section\ref{Methods} mentions different criteria, according to which survey papers were excluded from this work. However, they might still contain important information missed by excluding them. Specifically, certain works appeared to early\cite{AutomaticTestDataGeneration, BruteForceVulnDiscovery,BreakingSoftware}, or focus on symbolic execution based software testing, but do not primarily concern fuzzing\cite{DSETestGeneration, SurveySymbex, SearchStrategies, NewTrendsSymbex, ReviewConcolicTesting}.

  \subsubsection{Bibliometry}
  Survey papers might collectively be a good way to measure the importance of primary papers. By carefully selecting survey papers, looking at their bibliography, and counting how often each primary paper appears, one could get a measurement of importance for each. To ensure fairness, the scores would need to be weighted based on how many review papers were written after their publication date.

  Compared to examining the bibliographies of primary papers, works which introduced a new technique copied or adapted by many subsequent works would likely be less heavily weighted. This would skew the results to works highly influential based on their outcomes and ongoing development, not only the techniques they introduced.

  By doing time-based analysis, one might even be able to distinguish between works that are still relevant today, as opposed to works which were made redundant soon after their publication by other works implementing the same approach more successfully.

  Finally, analyzing the context, in which a certain citation appears might further improve accuracy of a bibliometric score of the importance of a primary work. If it is only mentioned, it could be deemed less important compared to if it is discussed extensively. Instead of complex linguistic analysis, counting how often a source is cited or examining the section they appear in might serve as an imperfect but easier to implement substitution.

  \subsubsection{Author Analysis}
  While reading the survey papers discussed in Section\ref{SurveyPapers}, one thing that became apparent is that some review papers were (co-)written by authors of important primary works. One example is Cristian Cadar, who is the author of EXE\cite{EXE}, KLEE\cite{KLEE}, KLEE-FP\cite{KLEEFP} RWset\cite{RWset}, KATCH\cite{KATCH}, EGT\cite{EGT}, Chopped\cite{Chopped}, and JFS\cite{JFS} and co-wrote review papers\cite{ReviewThreeDecades, ChallengesAndReflections, PreliminaryAssessment}.

  This is to be expected, since these authors already know the matter at hand very well. But it might also mean that their own work is over-represented in the review papers they helped assemble. By investigating this relationship, one might gain insight into how easy or hard it is to gain recognition as authors new to this field. It may also serve as a measure of how likely it is that innovative works by new authors are missed by review papers.

  \subsubsection{Genealogy}
  Since most works discussed in this paper introduce improvements in a specific part of a fuzzing system, they often rely on an existing system for the rest and extend it. One example is PYGMALION\cite{PYGMALION}, KATCH\cite{KATCH}, and Cloud9\cite{Cloud9} building on top of KLEE\cite{KLEE}, APLS\cite{APLS} extending SAGE\cite{SAGE}, or CRAXfuzz\cite{CRAXfuzz} extending S2E\cite{S2E}, which in turn again extends KLEE\cite{KLEE}.

  Building a genealogy tree from the primary works discussed might provide a list of highly adaptable and stable projects, which might in turn be better suitable for deployment in commercial applications. \citetitle{ArtScienceEng}\cite{ArtScienceEng} contains a diagram of such a genealogy. However, since this work does not focus on symbolic execution based fuzzing, it does not distinguish between nuances withing this subfield and only includes a limited amount of projects, while \citetitle{Hybrid}\cite{Hybrid} focuses on hybrid fuzzing systems only.

  Further, if not extending the code itself, ideas or concepts introduced by a paper might get copied or adapted by subsequent work. Building a genealogy dataset across these might uncover hidden relationships between projects and show the importance of ideas implemented in fuzzers across the board.

  \subsubsection{Meta-Survey}
  Finally, one could compare review papers based on different categories: primary works discussed (see above), categorizations of fuzzers, or more specifically categorization of problems that symbolic execution in fuzzing faces and the solutions to them that different works propose.

  \newpage
  \defbibheading{bibliography}[\bibname]{\section*{#1}}
  \addcontentsline{toc}{section}{\bibname}
  \printbibliography

  % \renewcommand{\thesubsection}{\arabic{subsection}}
  % \renewcommand{\thesubsubsection}{\arabic{subsection}.\arabic{subsubsection}}
  % \setcounter{subsection}{0}
  % \setcounter{subsubsection}{0}
  % \section*{Appendix}
  % \addcontentsline{toc}{section}{Appendix}
\end{multicols}

\end{document}
